%--------------------
% Packages
% -------------------
\documentclass[11pt,a4paper]{article}
\usepackage[utf8x]{inputenc}
%\usepackage{gentium}

\usepackage[pdftex]{graphicx} % Required for including pictures
\usepackage[pdftex,linkcolor=black,pdfborder={0 0 0}]{hyperref} % Format links for pdf
\usepackage{calc} % To reset the counter in the document after title page
\usepackage[english]{babel}     % for proper word breaking at line ends
\usepackage{graphicx}           % for ?
\usepackage{amsmath,amssymb}    % for better equations
\usepackage{amsthm}             % for better theorem styles
\usepackage{mathtools}          % for greek math symbol formatting
\usepackage{varwidth}
\usepackage{enumitem}           % for control of 'enumerate' numbering
\usepackage{listings}           % for control of 'itemize' spacing
 \setlength {\marginparwidth }{2cm}
\usepackage{todonotes}          % for clear TODO notes
\usepackage{pgfplots}
\usepackage{caption}
\usepackage{parskip}
\usepackage{makecell}
\usepackage{tabularx} 

\usepackage[a4paper, lmargin=0.1666\paperwidth, rmargin=0.1666\paperwidth, tmargin=0.1111\paperheight, bmargin=0.1111\paperheight]{geometry} %margins
%\usepackage{parskip}

\usepackage[all]{nowidow} % Tries to remove widows
\usepackage[protrusion=true,expansion=true,stretch=10]{microtype} % Improves typography, load after fontpackage is selected
\usepackage{hyperref}

\title{Research proposal master thesis}
\author{Julian van der Horst}

%-----------------------
% Begin document
%-----------------------
\begin{document} 
\maketitle
\section{Introduction}
This document will outline the research I will be doing for my master thesis. In general, I will do research into authenticated video calling in PubHubs. I will research the challenges and architectural designs needed to make such a feature, particularly looking into the security and privacy aspects of such a system and keeping them in line with the security and privacy considerations already made in PubHubs. 
\section{PubHubs}
Let me start by introducing PubHubs. PubHubs, short for Public Hubs, is an alternative community platform that puts public values first \cite{PH}. Where social media platforms like Instagram or Twitter make the individual central in their network, PubHubs makes a community central. The reasoning behind that is that this behavior closely resembles normal human interaction, where humans have small groups of friends with whom they interact. The way people experience social media now is very unnatural where if they engage with a platform it is shouted into the whole world.

"PubHubs is organized as a network of independent hubs, with a shared single-sign-on. Conversations take place in local hubs (and not globally) and the associated (conversation) data is managed decentrally, within each hub" \cite{PH}. Privacy is one of the core values of PubHubs, which can be seen in how it handles identity management. The main idea of Pubhubs is to allow anonymity but still have accountability and still have the ability to verify certain attributes of the user. The authentication of users is done using Yivi (formerly IRMA)\cite{YIVI} and the identity (the pseudonym) is generated using PEP \cite{PEP}.

\section{Video conferincing}
During the COVID-19 pandemic, there was a big need for remote work. Immediately services like Zoom \cite{Zoom} and Microsoft Teams \cite{MSTeams} became very popular. These tools allowed video conferencing between different remote locations in real-time. Video conferencing can be done in multiple ways but during this research, we will distinguish three types. Person-to-person (one-on-one), Person-to-person group, and speaker-to-audience. We distinguish these types because they all have certain properties that make their solution unique or possess some problems that will need to be resolved.

With video conferencing, people need to send and receive video streams, and they want to do this ideally in real time. A standard for that is WebRTC and it is now widely supported in browsers (97.77\%) \cite{CIUIWEBRTC}. WebRTC is both a protocol and an API that can be used for many things including video and audio communication. Matrix even has support for WebRTC and uses it in their main testing application Element \cite{ELEMENT}. 

To send this media to another person we have three ways of doing it. We either send the data in a P2P network. This means that the people in the call send their stream to every other person in the call. This makes it so that the server has no extra load since it does not handle any video streams. However, with every extra participant, every user needs to send/receive more and more data, at some point a bottleneck is reached and no extra participant can be added or a loss in data will occur. 

Another option would be to have a media server where the users would send their video streams and would get the incoming video streams. For such a media server we have two types: SFU (Selective forwarding unit) and MCU (Multipoint control unit). An SFU behaves in a way like a switch where it would selectively forward streams between clients, it does not interact with the video streams so it would work similarly with encrypted and unencrypted data.  

An MCU on the other hand takes in many media streams and combines them into one media stream. This makes it so that it can work very well with legacy systems since an MCU can receive various types of media and then output a standardized output. The MCU has to decode all the incoming media signals, which makes it costly at large scale since the CPU usage would be very high.

\section{Element call}
In PubHubs the communication is done via Matrix. "Matrix is an open source network for secure, decentralized communication" \cite{MATRIX}. The community can suggest changes to the Matrix protocol by submitting proposals, over the years this has added a lot of functionality to Matrix. In proposal 3401 \cite{MATRIX_VIDEO_CALL_PROP}, functionality for video calling was proposed. This proposal can be applied to all the above-mentioned ways of sending media data and describes a protocol for video calling. 

The same team that built matrix is also building Element \cite{ELEMENT}, which is an open-source implementation of a messaging app that uses matrix. A new feature was added to Element called Element call, which serves as an example implementation of the proposal mentioned above. They started by using peer-to-peer video calling since this more closely resembled the decentralized infrastructure of Matrix. However, this turned out to not be the best solution since it allowed for a maximum of 7 participants and video calling required lots of computer resources from participants. Recently they have released a new version that makes use of an open-source SFU called livekit \cite{LIVEKIT}.

We will use this implementation as a guide during our research, however, we will evaluate all the different choices made for Element call and deviate from them based on the requirements of PubHubs.

\section{Research question}
The main research question will be \"How to do authenticated video calls in PubHubs?\". This question then raises several sub-questions, questions like: What information should we allow to leak? (I.E. Who should be able to see what?). Should we allow users to see that other users are on a call? Should we allow users to call peer-to-peer and thus have no moderation? How to do encrypted video calls in PubHubs? Encryption can be done in multiple ways, do we want our users to have to share keys or should this be done via Yivi? 

While doing the research and writing the code for it we should also keep in mind the current development setup for PubHubs. This means that we should keep in mind which technologies they are using now and remain in those ecosystems. This makes it so that the code can be more easily maintained in the future. Learning this environment and adapting the code to their existing setup can be quite a challenge.

We can also extend the research into different areas that are not in the field of cyber security. We could do some research into the load of video calling on the server and see how this can be improved. Alternatively, research can be done into improving call quality by for instance adding noise reduction algorithms or background blurring. And lastly, we could research the usability of the video calling interface, where everything needs to be so that it makes sense to most users.

\bibliographystyle{plain}
\bibliography{main.bib}

\end{document}
